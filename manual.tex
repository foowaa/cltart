\documentclass[super,list,bibieee,myhdrone,table,math]{cltart}
\begin{document}
\cltheading{cltart使用说明by\ cltian}
\clttitle{cltart使用说明}
\cltinfo{cltian——tianchunlin123@gmail.com}
\section{cltart文档类简介}
\par ``cltart''是为CTEX的article类定制的一个文档类,用于简化常见中文文档书写。使用很简单:
\begin{verbatim}
\documentclass[arg1, arg2, ..., argN]{cltart}
\end{verbatim}
\par 具体的页面配置如下:
\begin{itemize}
  \item 使用A4页面,行间距1.5倍,段前段后0磅,section段前24磅、段后6磅,subsection段前12磅、段后6磅,subsection段前12磅、段后6磅
  \item 标题三号加黑黑体字,信息五号加黑黑体字
  \item 正文中文使用宋体小四,英文使用Times New Roman小四
  \item 页眉页脚使用五号字
  \item caption段前6磅,段后0磅,使用五号字
  \item 参考文献提供GBT-7714和IEEE Trans格式,有中文建议使用GBT-7714格式。文档使用五号字,1.5倍行距
  \item 对浮动体进行设置,使之排版更紧密
\end{itemize}
\section{内置命令}
\par 关于文档开头的命令。cltart摒弃了LaTex内置的生成标题、作者、日期等的命令。自建命令完成这些功能,这些命令可写可不写,不写则无此信息。其中,\verb|\clttitle{}|命令和\verb|\cltinfo{}|命令请注意先后顺序。
\begin{itemize}
  \item \verb|\clttitle{}|命令,用在开头,生成标题。如果不写则无标题。
  \item \verb|\cltinfo{}|命令,用在开头,生成相关信息(如作者,联系方式,日期等)。如果不写则无信息。
  \item \verb|\cltheading{}|命令,用在开头,如传入myhdrone, myhdrtwo,则必须设置,其为页眉信息。请参看节-\ref{sec3}。
\end{itemize}
\par 如本文开头是这样写的:
\begin{verbatim}
\documentclass[super,list,bibieee,myhdrone,color,tikz,table,math]{cltart}
\begin{document}
\cltheading{cltart使用说明by\ cltian}
\clttitle{cltart使用说明}
\cltinfo{cltian——tianchunlin123@gmail.com}
...
...
\end{document}
\end{verbatim}
\par 便捷的空格、换行、换页命令:
\begin{itemize}
  \item \verb|\smallblank{}|,可传入参数N,会生成N个1/3em的空格。
  \item \verb|\bigblank{}|,可传入参数N,会生成N个1em的空格。
  \item \verb|\nextline{}|,可传入参数N,会生换行N次。
  \item \verb|\blankpage{}|,可传入参数N,另起一页,并生成N个空白页。
  \item \verb|\nextpage|,无参数,另起一页。
\end{itemize}
\section{参考文献选项}
\par 如果要使用参考文献,请\textbf{务必}传入参数<numbers|super|authoryar>,如要使用IEEE Trans参考文献格式,使用参数bibieee,否则默认使用GBT-7714。传递参数方法如下,之后不再赘述。
\begin{verbatim}
\documentclass[super,bibieee]{cltart}
\end{verbatim}
\subsection{引用格式}\label{1.1}
\begin{itemize}
  \item numbers:[1]
  \item super:上标 [1]
  \item authoryear:(Jones, 1995)
\end{itemize}
\par 引用使用\verb|\cite{}|,例如\smallblank{1}如文献\cite{tian2017deep,yuan2018auxiliary}。如要在super中使用numbers,使用命令\verb|\citens{}|,例如\smallblank{1}如文献\citens{tian2017deep,yuan2018auxiliary}
\subsection{参考文献格式}
\par 默认使用GBT-7714格式,如想使用IEEE Trans格式,请传入bibieee参数。注意:IEEE Trans格式比较适用于参考文献为英文论文,否则不要使用。
\section{页眉页脚}\label{sec3}
采用fancyhdr包,我们预定义了6种页眉页脚格式,分别为:myhdrone,myhdrtwo,myhdrthree,myhdrfour,myhdrfive和默认,不设置即为默认格式。
\begin{itemize}
  \item myhdrone——必须设置\verb|\cltheading{}|,页眉左侧为\verb|\cltheading{}|的内容,右侧为页码。
  \item myhdrtwo——必须设置\verb|\cltheading{}|,页眉居中为\verb|\cltheading{}|的内容,页脚居中为页码。
  \item myhdrthree——页眉左侧为章节号和章节标题,右侧为页码。
  \item myhdrfour——页眉居中为章节号和章节标题,页脚居中为页码。
  \item myhdrfive——页眉页脚为空。
  \item 默认——页眉为空,页脚居中为页码。
\end{itemize}
\section{杂项}
\par \textbf{list}。引入verbatim,listing,salgpseudocode,algorithm,algorithmicx包,并进行了一些配置,可以支持verbatim模式,源代码和伪代码。源代码如下:
\begin{lstlisting}[language=c++]
typedef struct ImageData {
  ImageData() {
    data = nullptr;
    width = 0;
    height = 0;
    num_channels = 0;
  }

  ImageData(int32_t img_width, int32_t img_height,
    int32_t img_num_channels = 1) {
    data = nullptr;
    width = img_width;
    height = img_height;
    num_channels = img_num_channels;
  }

  uint8_t* data;
  int32_t width;
  int32_t height;
  int32_t num_channels;
} ImageData;

typedef struct Rect {
  int32_t x;
  int32_t y;
  int32_t width;
  int32_t height;
} Rect;

typedef struct FaceInfo {
  seeta::Rect bbox;
  double roll;
  double pitch;
  double yaw;
  double score;
} FaceInfo;

  typedef struct {
    double x;
    double y;
  } FacialLandmark;
}
\end{lstlisting}
\par \textbf{math}。引入了amsmath,mathtools,amsfonts,amssymb方便高级数学公式书写。例如:
\nextline{1}

Pascal’s rule is
\[
\binom{n}{k} =\binom{n-1}{k}
+ \binom{n-1}{k-1}
\]

\par \textbf{color}。引入了xcolor包。具体使用参见\url{https://en.wikibooks.org/wiki/LaTeX/Colors},其相当于:
\begin{verbatim}
\usepackage[usenames,dvipsnames,table]{xcolor}
\end{verbatim}

\par \textbf{table}。引入了ctable,longtable,multirow包,方便进行复杂表格设计。例子如表-\ref{table1}。

\ctable[cap = {TABLE CAPTION},caption = {例子}, label={table1}]{cc}
{
  % You specify table footnotes here.
  \tnote[$\ast$]{DA FOOTNOTE 1}
  \tnote[$\dagger$]{dat other footnote}
  \tnote[b]{mistakes are possible (you must match these up yourself)}
}{
\FL % FLORIDA (just kidding, means "first line")
COL 1\tmark[a] & COL 2\tmark[$\ast$]
\ML % middle line
6.920e+00\tmark[$\dagger$]     &   0.09781\\
97     &   2000
\LL % last line
}

\par \textbf{nohref}。如传入nohref参数,则不使用超链接。默认使用。

\par \textbf{geometry}。引入geomrtry包调整页面。

\par \textbf{tikz}。引入tikz包绘图。

\par \textbf{syntaxonly}。引入syntonly包,进行语法检查且不生成pdf文件(往往这样会生成文件更快),此选项比较适合于确认文档有无语法错误。

\par \textbf{图片}。默认导入graphicx,subcaption,bicaption包,并进行了图片样式的调整。详情使用见\url{https://en.wikibooks.org/wiki/LaTeX/Importing_Graphics}和\url{https://en.wikibooks.org/wiki/LaTeX/Floats,_Figures_and_Captions}。
\bibliography{bib/ref}
\end{document}
